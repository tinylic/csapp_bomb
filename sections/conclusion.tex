\section{conclusion}

	这次的Bomb Lab总计花了我7个小时,从晚上6点拆到凌晨1点,才算大功告成。不可否认,这个lab对我对于汇编的理解起到了很强的巩固作用。从一开始面对冗长汇编代码的手足无措,到一步步将炸弹抽丝剥茧,慢慢拆弹,到最后将隐藏关解决,这其中的每一关对我都是新的挑战。
	
	这次lab的覆盖范围非常广泛:第一关的函数调用,第二关的数组的存储,第三关的switch跳转表的实现。。几乎每一关的知识点都不重合,设计者的用心良苦可见一斑。
	
	同时,通过这次lab, 我也系统学习了gdb的使用。在平常的编程环境中,我们常常只需要在IDE中用鼠标设置断点,点开watch窗口,就可以轻松享用IDE的调试功能,却很少想过这些调试功能都是怎么实现的。通过这次一步步的手动输入gdb命令,我对于IDE的调试功能有了一个粗浅的了解。当然,对于主流IDE调试功能的更深层次的实现,是我接下来需要探索的。
	
	陆游说过:“纸上得来终觉浅,绝知此事要躬行。” 对于CSAPP这门课来说,更是如此。通过这次lab,通过大量的读代码,gdb调试,我对于代码的底层实现有了新的认识。我想,这就是老师布置这次lab的用意之一吧。
