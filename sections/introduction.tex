\section{introduction}
	\subsection{Overview}
	A "binary bomb" is a program provided to students as an object code file. When run, it prompts the user to type in 6 different strings. If any of these is incorrect, the bomb ``explodes,'' printing an error message and logging the event on a grading server. Students must ``defuse'' their own unique bomb by disassembling and reverse engineering the program to determine what the 6 strings should be. The lab teaches students to understand assembly language, and also forces them to learn how to use a debugger. It's also great fun. A legendary lab among the CMU undergrads.
	\subsection{Preparation}
	在拆除炸弹之前,首先进行一些前期准备:
	\begin{itemize}
		\item	通过putty登陆服务器,发现我本次Lab的实验文件:\textbf{bomb51.tar}
		\item	执行\textbf{tar xvf bomb51.tar}解压文件,得到四个文件:bomb bomb.c ID README
		\item	其中 ID README 分别是学生的编号和这次的说明文档。
		
				于是查看\textbf{bomb.c}, 发现其头文件声明如下:
				\lstinputlisting{sources/header.txt}
				
				phase.h包含了这次炸弹的全部关卡,而它并没有在Lab中给出。因此我们只能从可执行文件\textbf{bomb}下手
		\item	执行\textbf{objdump -d bomb > bomb.txt}, 反编译可执行文件,将汇编代码输出到\textbf{bomb.txt}
		\item	反编译之后的代码非常长,不过在仔细研究之后,发现其中有六个函数\textbf{<phase$\_$1..6>},分别对应六个关卡。因此,这次Lab的关键就是破解这六个关卡对应的汇编代码,分析这六个函数的功能。
	\end{itemize}